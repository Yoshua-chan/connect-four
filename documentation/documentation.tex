\documentclass{book} 
% \usepackage{polski}
\usepackage[utf8]{inputenc}

% \usepackage[QX]{fontenc}
% \usepackage{lmodern}


\title{Connect Four}
\author{Kacper Filipek}
\date{\today}

\begin{document}

\maketitle

\section{Opis projektu}

Projekt jest realizacją gry w "Czwórki" (ang. "Connect Four") w C++, z interfejsem TUI (t.j. interfejs używający znaków specjalnych i kolorów do rysowania interfejsu użytkownika w konsoli).


\section{Project description}

The project is a realisation of the game "Connect Four" made in C++ with TUI interface (an interface using special characters for drawing UI in the terminal)

\section{Instrukcja użytkownika}

W tym punkcie należy umieścić instrukcję użytkowania programu. 
Może być to na przykład opis poszczególnych menu w programie. 
W przypadku gry należy opisać zasady gry. 
Opcjonalnie można wstawić zrzuty ekranu. 
Jeśli uruchomienie programu wymaga wykonania jakiś 
niestandardowych lub dodatkowych czynności 
(na przykład uruchomienie serwera baz danych itp.) 
to należy zamieścić tę informację.

\section{Kompilacja}

Program został napisany na systemy operacyjne z rodziny Linux, 
chociaż powinien on działać na Windowsie, ponieważ kod nie 
używa żadnych zależnych od platformy plików nagłówkowych. 
Program używa  systemu CMake do budowania projektu. 
Można skompilować go na dwa sposoby:
    \begin{itemize}
         
    \item Sposób 1:
        \begin{enumerate}
            \item Wejść do folderu build/
            \item Wykonać polecenie cmake ..
        \end{enumerate}

    \item Sposób 2:
        \begin{enumerate}
            \item Z katalogu głównego wywołać skrypt ./bld.sh. Ze względu na fakt, że skrypt wywołuje program make, może on nie działać na Windowsie.
        \end{enumerate}

    \end{itemize}

Po zbudowaniu powienien się plik wykonywalny o ./build/connect-four. 
Z uwagi na fakt, że ścieżki do zasobów są wpisane w programie relatywnie 
do głównego katalogu, to program wykonywalny powinien z niego wywoływany.

\section{Pliki źródłowe}

\section{Zależności}

\section{Opis klas}

\section{Zasoby}

\section{Dalszy rozwój i ulepszenia}

\section{Inne}


\end{document}

